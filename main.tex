% !TEX TS-program = xelatex% !TEX encoding = UTF-8
\documentclass{aas}
\usepackage{multicol}
\usepackage{subfigure}
\usepackage{amsmath}
\usepackage{amssymb}
\usepackage{amsfonts}
\usepackage{graphicx}
\usepackage{url}
\usepackage{ccaption}
\usepackage{booktabs} % 做三线表的上下两条粗线用

\setcounter{page}{1}

\begin{document}

\cntitle{{\hei\qquad 《自动化学报》稿件加工样本}
	
\thanks{收稿日期\
XXXX-XX-XX
\quad
录用日期\
XXXX-XX-XX}

\thanks{Manuscript received
Month Date, Year;
accepted
Month Date, Year}

\thanks{国家重点基础研究发展计划(973计划) (XXXXXX),国家高技术研究发展计划(863计划) (XXXXXX),国家自然科学基金(XXXXXX)资助}

\thanks{Supported by National Basic Research Program of China (973 Program) (XXXXXX),
National High Technology Research and Development Program of China (863 Program) (XXXXXX),
National Natural Science Foundation of China (XXXXXX)}

\thanks{本文责任编委\ XXX}

\thanks{Recommended by Associate Editor BIAN Wei}

\thanks{1.
中国科学院自动化研究所高技术创新中心\ 北京\ 100190
\quad 2.
中国科学院自动化研究所模式识别国家重点实验室\ 北京\ 100190
\quad 3.
中国科学院自动化研究所《国际自动化与计算杂志》编辑部\ 北京\ 100190
\quad 4. 中国科学院自动化研究所《自动化学报(英文版)》编辑部\ 北京\ 100190
\quad 5. 中国科学院自动化研究所《自动化学报》编辑部\ 北京\ 100190}

\thanks{1.
Hi-Tech Innovation Center, Institute of Automation, Chinese Academy of Sciences, Beijing
100190
\quad 2.
National Laboratory of Pattern Recognition,
Institute of Automation, Chinese Academy of Sciences, Beijing 100190
\quad 3.
Editorial
Office of {\sl International Journal of Automation and Computing},
Institute of Automation, Chinese Academy of Sciences, Beijing 100190
\quad 4. Editorial
Office of {\sl IEEE/CAA Journal of Automatica Sinica (JAS)}, Institute of Automation,
Chinese Academy of Sciences, Beijing 100190
\quad 5. Editorial
Office of {\sl Acta Automatica Sinica}, Institute of Automation,
Chinese Academy of Sciences, Beijing 100190
}}

\cnauthor{尚书林$^{\scriptscriptstyle1,\,2}$
\hspace{1em}
左年明$^{\scriptscriptstyle2}$
\hspace{1em}
陈培颖$^{\scriptscriptstyle3}$
\hspace{1em}
欧\ 彦$^{\scriptscriptstyle4,\,5}$
\hspace{1em}
张\ 哲$^{\scriptscriptstyle4,\,5}$
}

\cnabstract{中文摘要字数应为文章总字数的5\,\%左右,一般不超过200字.摘要应涵盖全文.摘要内容包括研究目的、方法、结果等,注意不是标题的罗列,能独立成文,不能出现公式号和文献号.}

\cnkeyword{关键词1,关键词2,关键词3,关键词4,关键词5}

\doi{10.16383/j.aas.20xx.cxxxxxx}

\entitle{Preparation of Papers for Acta Automatica Sinica}

\enauthor{SHANG Shu-Lin$^{1,\,2}$
\qquad
ZUO Nian-Ming$^2$
\qquad
CHEN Pei-Ying$^3$
\qquad
OU Yan$^{4,\,5}$
\qquad
ZHANG Zhe$^{4,\,5}$
}

\enabstract{An abstract should be a concise summary of the
significant items in the paper, including the results and
conclusions. It should be about 5\,\% of the length of the article,
but not more than 200 words. Define all nonstandard symbols,
abbreviations and acronyms used in the abstract. Do not cite
references in the abstract.}

\enkeyword{Keyword 1, keyword 2, keyword 3, keyword 4, keyword 5}

\cnaddress{尚书林,左年明,陈培颖,欧彦,张哲.
《自动化学报》稿件加工样本. 自动化学报, 201X,
\textbf{XX}(X): X$-$X}

\enaddress{Shang Shu-Lin, Zuo Nian-Ming, Chen Pei-Ying, Ou Yan, Zhang Zhe.
Preparation of papers for Acta Automatica Sinica.
\textsl{Acta Automatica Sinica}, 201X, \textbf{XX}(X): X$-$X}

\maketitle

\pagestyle{aasheadings}

\section{引言}
本文是《自动化学报》中文稿件\LaTeX 模版的一个样例和使用说明. 模版的中文支持部分采用CCT. 这一章为引言, 无需写标题.
这是一个引用\cite{ran1996modeling}.

\section{方法}
\subsection{第一种方法}
这是第一种方法的描述。
\subsection{第二种方法}
这是第二种方法的描述。

\section{结论}
本文得出了以下结论...



\bibliographystyle{unsrt}
\bibliography{refs}


\begin{biography}[Image/ssl.eps]
\noindent{\hei
尚书林
}\quad
中国科学院自动化研究所博士研究生.
2002年获得北京师范大学信息学院电子系学士学位.
主要研究方向为图像与视频压缩技术.\\E-mail: aas\_editor@ia.ac.cn

\noindent({\bf
SHANG Shu-Lin
}\quad
Ph.\,D. candidate at the
Institute of Automation, Chinese Academy of Sciences. He received
his bachelor degree from Beijing Normal University in 2002. His research
interest covers image compression and video coding.)
\end{biography}

\begin{biography}[Image/nmzuo.eps]
\noindent{\hei
左年明
}\quad
中国科学院自动化研究所博士研究生.
2002年获得山东大学数学学院学士学位.
主要研究方向为医学图像处理, CT图像重建.\\E-mail: aas\_editor@ia.ac.cn

\noindent({\bf
ZUO Nian-Ming
}\quad
Ph.\,D. candidate at the
Institute of Automation, Chinese Academy of Sciences. He received
his bachelor degree from Shandong University in 2002. His research
interest covers medical CT image reconstruction and medical image
processing.)
\end{biography}

\begin{biographynophoto}
\noindent{\hei 陈培颖}\quad 《国际自动化与计算杂志》编辑部责任编辑.\\E-mail: peiying.chen@ia.ac.cn

\noindent({\bf CHEN Pei-Ying}\quad Managing editor at the Editorial Office of
{\sl International Journal of Automation and Computing}.)
\end{biographynophoto}

\begin{biographynophoto}
\noindent{\hei 欧\hskip2.5mm彦}\quad 《自动化学报(英文版)》编辑部责任编辑.\\E-mail: yan.ou@ia.ac.cn

\noindent({\bf OU Yan}\quad Managing editor at the Editorial Office of
{\sl IEEE/CAA Journal of Automatica Sinica (JAS)}.)
\end{biographynophoto}

\begin{biography}[Image/zz.eps]
\noindent{\hei
张\hskip2.5mm哲
}\quad
《国际自动化与计算杂志》与《自动化学报》编辑部技术编辑. 2006年获得北京工业大学电控学院自动化系学士学位.
本文通信作者.\\E-mail: zhe.zhang@ia.ac.cn

\noindent({\bf
ZHANG Zhe
}\quad
Copy-editor at the Editorial Office of
{\sl International Journal of Automation and Computing}, and {\sl
Acta Automatica Sinica}. He received his bachelor degree from Beijing
University of Technology in 2006. Corresponding author of this paper.)
\end{biography}

\end{document}
